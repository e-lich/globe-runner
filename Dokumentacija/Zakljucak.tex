\chapter{Zaključak i budući rad}
		 
		 Naša je grupa imala zadatak razviti web aplikaciju koja funkcionira kao online igra u kojoj igrači prikupljaju kartice i s njima ulaze u bitke s obližnjim igračima, dok najbolji među njima mogu predlagati nove lokacije koje kartografi pak pregledavaju, uređuju i odobravaju, a administratori sve to nadziru uz potpunu kontrolu nad korisnicima i lokacijama. Na ovom smo projektu radili 10 tjedana i uspješno smo implementirali tražene funkcionalnosti.

        Tijekom prvog nastavnog ciklusa oformili smo tim i postavili neke organizacijske premise, a zatim se i podijelili u  \textit{frontend} i \textit{bakcend} podtimove, uz dio članova posebno fokusiran na izradu dokumentacije projekta. Značajnu smo količinu truda i vremena uložili u dokumentiranje zahtjeva i pripremu kostura projekta, koji su nam naposlijetku bili uvelike od koristi.

        U drugom nastavnom ciklusu intenzivno smo radili na implementaciji dokumentiranih funkcionalnosti i ključni pristup koji nam je pomogao u radu bilo je timsko uhodavanje u tehnologije koje je praćeno i samostalnim radom članova. Isprva smo značajnu količinu vremena trošili na organizaciju i podjelu posla, no ubrzo smo tome doskočili organizacijskom tablicom i direktnom komunikacijom članova backend i frontend timova. Problem na koji smo naišli na kraju jest nemogućnost korištenja aplikacije na udaljenom servisu jer korištenje lokacije zahtijeva komunikaciju HTTPS protokolom, no to nismo uspjeli postići dodavanjem certifikata na korištenom servisu.

        Komunikaciju unutar tima ostvarili smo na nekoliko razina -  WhatsApp grupu koristili smo za bitne obavijesti i dogovore oko općih sastanaka, vlastiti Discord server koristili smo za odvajanje pisane i glasovne komunikacije unutar tima na teme general, backend, frontend i dokumentacija. Ti su nam kanali služili za svakodnevnu suradnju i smanjenje spam sadržaja. Uz navedeno, koristili smo i zajednički Google disk gdje smo vodili tjedne bilješke sastanaka koje su nam služile za isticanje tekuće problematike i evidentiranje pomaka i resursa, kao i ideja koje smo svi istovremeno bilježili tijekom sastanaka.

        Rad na ovom projektu na razne je načine bio novi za sve nas, posebno zbog zahtijevane koordinacije i balansiranja zadataka među članovima. Nailazili smo na manje poteškoće u komunikaciji i međusobnim očekivanjima no uspješno smo prebrodili i te izazove. U realizaciji projekta mnogo bismo toga promijenili, no zbog drugih opterećenja tijekom semestra trudili smo se postići osnovnu funkcionalnost aplikacije i u tome smo uspjeli te smo zato zadovoljni ukupnom provedbom projekta.
		
		\eject 