\chapter{Dnevnik promjena dokumentacije}
		
		\textbf{\textit{Kontinuirano osvježavanje}}\\
				
		
		\begin{longtblr}[
				label=none
			]{
				width = \textwidth, 
				colspec={|X[2]|X[13]|X[3]|X[3]|}, 
				rowhead = 1
			}
			\hline
			\textbf{Rev.}	& \textbf{Opis promjene/dodatka} & \textbf{Autori} & \textbf{Datum}\\[3pt] \hline
			0.1 & Popunjen predložak za naslovnicu.	& V. Pezo & 29.10.2022. 		\\[3pt] \hline 
			0.2	& Napisana prva verzija opisa projekta. & V. Pezo & 30.10.2022. 	\\[3pt] \hline 
			0.3 & Funkcionalni zahtjevi, prvi dio obrazaca uporabe, ostali zahtjevi. & M. Šola & 02.11.2022. \\[3pt] \hline 
			0.4	& Raspisani obrasci uporabe do kraja & V. Pezo & 2.11.2022. 	\\[3pt] \hline 
			0.5	& Dodani dijagrami obrazaca upotrebe. & L. Raguž & 3.11.2022. 	\\[3pt] \hline 
			0.6	& Dodani sekvencijski dijagrami & E. Kumer & 3.11.2022. 	\\[3pt] \hline
			0.7 & Ispravci 2. i 3. poglavlja, dodavanje fotografije. & V. Pezo & 16.11.2022. \\[3pt] \hline 
			0.8 & Napisan opis arhitekture sustava, opis baze podataka & M. Šola & 17.11.2022. \\[3pt] \hline 
			0.9 & Izmijenjeni sekvencijski dijagrami, dodan opis sekvencijskih dijagrama i zapisnici sastanaka & E. Kumer & 17.11.2022. \\[3pt] \hline 
			0.10 & Dodan opis dijagrama razreda i dijagram za Models & V. Pezo & 18.11.2022. \\[3pt] \hline 
			0.11 & Dodani dijagrami Controllers i Data transfer object u dijelu s dijagramima razreda & L. Raguž & 18.11.2022. \\[3pt] \hline 
			1.0 & Konačni ispravci, prijelomi stranica, aktivnost & V. Pezo & 18.11.2022. \\[3pt] \hline 
		
		\end{longtblr}
	
	
	%	\textit{Moraju postojati glavne revizije dokumenata 1.0 i 2.0 na kraju prvog i drugog ciklusa. Između tih revizija mogu postojati manje revizije već prema tome kako se dokument bude nadopunjavao. Očekuje se da nakon svake značajnije promjene (dodatka, izmjene, uklanjanja dijelova teksta i popratnih grafičkih sadržaja) dokumenta se to zabilježi kao revizija. Npr., revizije unutar prvog ciklusa će imati oznake 0.1, 0.2, …, 0.9, 0.10, 0.11.. sve do konačne revizije prvog ciklusa 1.0. U drugom ciklusu se nastavlja s revizijama 1.1, 1.2, itd.}