\chapter{Implementacija i korisničko sučelje}
		
		
		\section{Korištene tehnologije i alati}
		
			 
			 Tim je za potrebe komunikacije koristio \underline{WhatsApp}\footnote{https://www.whatsapp.com/} aplikaciju i \underline{Discord}\footnote{https://discord.com/} platformu. WhatsApp je služio za osnovnu komunikaciju poput dogovora sastanaka i prenošenje najbitnijih informacija o napretku projekta. Za online sastanke se koristio Discord jer je omogućavao jednostavno spajanje u glasovni poziv i dijeljenje ekrana drugim članovima tima, a uz to nudi i mogućnost organiziranja različitih kanala kako bi komunikacija među podtimovima (\textit{frotend}, \textit{backend}) bila organizirana. Korišteni sustav za upravljanje izvornim kodom je \underline{Git}\footnote{https://git-scm.com/}, a udaljeni repozitorij je postavljen na web platformu \underline{GitLab}\footnote{https://about.gitlab.com/}. \par
			 
			 Za pisanje dokumentacije je bio korišten \underline{Overleaf}\footnote{https://www.overleaf.com/} što je cloud-based LaTeX editor zahvaljujući kojem je dokumentacija s aktualnim promjenama bila dostupna svakom članu bez dodatnih instalacija. Kao provjeru pravopisa koristili smo online alat \underline{ispravi.me}\footnote{https://ispravi.me/}. UML dijagrame smo izrađivali u alatu \underline{Astah UML}\footnote{https://astah.net/products/astah-uml/}, a dijagram baze podataka je bio izrađen u alatu \underline{dbdiagram.io}\footnote{https://dbdiagram.io/home}. \par
			 
			 Odabrano razvojno okruženje za našu aplikaciju bio je \underline{Visual Studio Code}\footnote{https://code.visualstudio.com/} koji ima integrirani terminal i podršku za Git te uključuje podršku za debugiranje. VS Code je odabran zbog toga što je dostupan za sve operacijske sustave koje smo koristili za rad na projektu - Windows, Linux i macOS. \par
			 
            U pisanju aplikacije je korišten \underline{Flask}\footnote{https://flask.palletsprojects.com/en/2.2.x/} kao radni okvir za programski jezik \underline{Python}\footnote{https://www.python.org/}. Uz to je korišten i alat \underline{pipenv}\footnote{https://pipenv.pypa.io/en/latest/} koji pomaže pri stvaranju virtualnog okruženja za \textit{backend} dio aplikacije te alat \underline{SQLAlchemy}\footnote{https://www.sqlalchemy.org/} koji olakšava korištenje baze podataka i SQL-a u \textit{backendu}. Za izradu \textit{frotenda} se koristio programski jezik \underline{TypeScript}\footnote{https://www.typescriptlang.org/} koji je zapravo omotač za programski jezik \underline{JavaScript}\footnote{https://www.javascript.com/} te ima dodatne funkcionalnosti u odnosu na JavaScript. Također, koristio se \underline{React}\footnote{https://reactjs.org/} koji je JavaScript library, ali ga se često smatra i radnim okvirom, iako to tehnički nije. \underline{Material UI}\footnote{https://mui.com/}, tj. MUI je jedan od libraryja koji se uvelike koristi u projektu zbog svoje jednostavnosti i povezanosti s Reactom. On nudi i omogućava korištenje već gotovih komponenti te olakšava stvaranje novih. \par
            
            Za prikazivanje mapa u aplikaciji koristio se \underline{Leaflet}\footnote{https://leafletjs.com/}, JavaScript library za interaktivne mape, a za generiranje ruta kartografa koristio se \underline{OSRM}\footnote{https://project-osrm.org/}, open source alat koji generira najkraću rutu za obilazak željenih točaka.
            
            Baza podataka se nalazi na cloud poslužitelju \underline{Render}\footnote{https://render.com/}, a \textit{frotend} i \textit{backend} su postavljeni na web poslužitelju \underline{DigitalOcean}\footnote{https://www.digitalocean.com/}. \par
			
			
			\eject 
		
	
		\section{Ispitivanje programskog rješenja}
			
			\textbf{\textit{dio 2. revizije}}\\
			
			 \textit{U ovom poglavlju je potrebno opisati provedbu ispitivanja implementiranih funkcionalnosti na razini komponenti i na razini cijelog sustava s prikazom odabranih ispitnih slučajeva. Studenti trebaju ispitati temeljnu funkcionalnost i rubne uvjete.}
	
			
			\subsection{Ispitivanje komponenti}
			\textit{Potrebno je provesti ispitivanje jedinica (engl. unit testing) nad razredima koji implementiraju temeljne funkcionalnosti. Razraditi \textbf{minimalno 6 ispitnih slučajeva} u kojima će se ispitati redovni slučajevi, rubni uvjeti te izazivanje pogreške (engl. exception throwing). Poželjno je stvoriti i ispitni slučaj koji koristi funkcionalnosti koje nisu implementirane. Potrebno je priložiti izvorni kôd svih ispitnih slučajeva te prikaz rezultata izvođenja ispita u razvojnom okruženju (prolaz/pad ispita). }
			
			
			
			\subsection{Ispitivanje sustava}
			
			 \textit{Potrebno je provesti i opisati ispitivanje sustava koristeći radni okvir Selenium\footnote{\url{https://www.seleniumhq.org/}}. Razraditi \textbf{minimalno 4 ispitna slučaja} u kojima će se ispitati redovni slučajevi, rubni uvjeti te poziv funkcionalnosti koja nije implementirana/izaziva pogrešku kako bi se vidjelo na koji način sustav reagira kada nešto nije u potpunosti ostvareno. Ispitni slučaj se treba sastojati od ulaza (npr. korisničko ime i lozinka), očekivanog izlaza ili rezultata, koraka ispitivanja i dobivenog izlaza ili rezultata.\\ }
			 
			 \textit{Izradu ispitnih slučajeva pomoću radnog okvira Selenium moguće je provesti pomoću jednog od sljedeća dva alata:}
			 \begin{itemize}
			 	\item \textit{dodatak za preglednik \textbf{Selenium IDE} - snimanje korisnikovih akcija radi automatskog ponavljanja ispita	}
			 	\item \textit{\textbf{Selenium WebDriver} - podrška za pisanje ispita u jezicima Java, C\#, PHP koristeći posebno programsko sučelje.}
			 \end{itemize}
		 	\textit{Detalji o korištenju alata Selenium bit će prikazani na posebnom predavanju tijekom semestra.}
			
			\eject 
		
		
		\section{Dijagram razmještaja}
			
%	\textbf{\textit{dio 2. revizije}}\\
			
%	\textit{Potrebno je umetnuti \textbf{specifikacijski} dijagram razmještaja i opisati ga. Moguće je umjesto specifikacijskog dijagrama razmještaja umetnuti dijagram razmještaja instanci, pod uvjetom da taj dijagram bolje opisuje neki važniji dio sustava.}\\
			
			Na slici \ref{fig:DeploymentDiagram} prikazan je specifikacijski dijagram razmještaja. Iz njega lako možemo iščitati topologiju našeg sustava te odnos između sklopovskih i programskih komponenti. Kao što vidimo, korisnici (administratori, igrači i kartografi) internetskim preglednikom pristupaju web-aplikaciji. Komunikacija između korisničkog i poslužiteljskog računala odvija se preko HTTP veze. Web browser šalje HTTP serveru \textit{HTTP request} na port 3000. HTTP server ovisan je o web-poslužitelju aplikacije, u našem slučaju DigitalOcean, na kojem se nalazi GlobeRunner. Svi potrebni podaci za ispravan rad aplikacije spremaju se u globe-runner bazu podataka koja je "deployana" na Render poslužitelju. Dohvaćanje tih podataka odvija se tako da aplikacija šalje \textit{Data request} na port 5432. Također, u slučaju registracije novog korisnika (igrača ili kartografa) aplikacija SMTP protokolom šalje email s linkom za potvrdu kreiranja profila. Pretpostavljamo da tada korisnici taj isti mail primaju IMAP protokolom dok se spajaju na poslužitelja e-pošte.

			 \begin{figure}[H]
        			\includegraphics[width=\textwidth]{slike/deploymentdiagram.png}
        			\centering
        			\caption{Dijagram razmještaja}
        			\label{fig:DeploymentDiagram}
        		\end{figure}
			
			\eject 
		
		\section{Upute za puštanje u pogon}
		
			\textbf{\textit{dio 2. revizije}}\\
		
			 \textit{U ovom poglavlju potrebno je dati upute za puštanje u pogon (engl. deployment) ostvarene aplikacije. Na primjer, za web aplikacije, opisati postupak kojim se od izvornog kôda dolazi do potpuno postavljene baze podataka i poslužitelja koji odgovara na upite korisnika. Za mobilnu aplikaciju, postupak kojim se aplikacija izgradi, te postavi na neku od trgovina. Za stolnu (engl. desktop) aplikaciju, postupak kojim se aplikacija instalira na računalo. Ukoliko mobilne i stolne aplikacije komuniciraju s poslužiteljem i/ili bazom podataka, opisati i postupak njihovog postavljanja. Pri izradi uputa preporučuje se \textbf{naglasiti korake instalacije uporabom natuknica} te koristiti što je više moguće \textbf{slike ekrana} (engl. screenshots) kako bi upute bile jasne i jednostavne za slijediti.}
			
			
			 \textit{Dovršenu aplikaciju potrebno je pokrenuti na javno dostupnom poslužitelju. Studentima se preporuča korištenje neke od sljedećih besplatnih usluga: \href{https://aws.amazon.com/}{Amazon AWS}, \href{https://azure.microsoft.com/en-us/}{Microsoft Azure} ili \href{https://www.heroku.com/}{Heroku}. Mobilne aplikacije trebaju biti objavljene na F-Droid, Google Play ili Amazon App trgovini.}
			
			
			\eject 