\chapter{Arhitektura i dizajn sustava}
		
		Arhitektura se može podijeliti na tri podsustava:
		    \begin{packed_item}
		        \item Web poslužitelj
    		    \item Web aplikacija
    		    \item Baza podataka
		    \end{packed_item}
		    
	    \underline{\textit{Web preglednik}} je program koji korisniku omogućuje pregled web-stranica i multimedijalnih sadržaja vezanih uz njih. Svaki internetski preglednik je prevoditelj što znači da preglednik interpretira kod kojim je stranica napisana kao nešto razumljivo. Korisnik putem web preglednika šalje zahtjev web poslužitelju.
	    
	    \underline{\textit{Web poslužitelj}} je osnova rada web aplikacije. Njegova primarna zadaća je komunikacija klijenta s aplikacijom. Komunikacija se odvija preko HTTP (engl. \textit{Hyper Text Transfer Protocol}) protokola, što je protokol u prijenosu informacija na webu. Poslužitelj je onaj koji pokreće web aplikaciju te joj prosljeđuje zahtjev.

	    Korisnik koristi \underline{\textit{web aplikaciju}} za obrađivanje željenih zahtijeva. Web aplikacija obrađuje zahtjev te ovisno o zahtjevu, pristupa bazi podataka nakon čega preko poslužitelja vraća korisniku odgovor u obliku HTML dokumenta vidljivog u web pregledniku.
		
		Programski jezik kojeg smo odabrali za izradu naše web aplikacije je Python zajedno s Flask radnim okvirom te programski jezik TypeScript s React radnim okvirom. Odabrano razvojno okruženje je Microsoft visual studio. Arhitektura sustava temeljit će se na MVC (Model-View-Controller) konceptu. MVC koncept podržan je od strane Flask radnog okvira i kao takav ima gotove predloške koji nam olakšavaju razvoj razvoj web aplikacije.
		
        Karakteristika MVC koncepta je nezavisan razvoj pojedinih dijelova aplikacije što za posljedicu ima jednostavnije ispitivanje kao i jednostavno razvijanje i dodavanje novih svojstava u sustav.
        MVC koncept sastoji se od:
        \begin{packed_item}
            \item \textbf{Model} – središnja komponenta sustava. Predstavlja dinamičke strukture podataka, neovisno o korisničkom sučelju. Izravno upravlja podacima, logikom i pravilima aplikacije te prima ulazne podatke od Controllera
            \item \textbf{View} – bilo kakav prikaz podataka, poput grafa. Mogući su različiti prikazi iste informacije poput grafičkog ili tabličnog prikaza podataka.
            \item \textbf{Controller} – prima ulaze i prilagođava ih za prosljeđivanje Modelu ili Viewu. Upravlja korisničkim zahtjevima i temeljem njih izvodi daljnju interakciju s ostalim elementima sustava.
        \end{packed_item}

				
		\section{Baza podataka}
			
			Za potrebe našeg sustava koristit ćemo relacijsku bazu podataka koja svojom strukturom olakšava modeliranje stvarnog svijeta. Gradivna jedinka baze je relacija, odnosno tablica koja je definirana svojim imenom i skupom atributa. Zadaća baze podataka je brza i jednostavna pohrana, izmjena i dohvat podataka za daljnju obradu. Baza podataka ove aplikacije sastoji se od sljedećih entiteta:
			\begin{packed_item}
			    \item User
                \item Cartographer
                \item Player
                \item Admin
                \item Card
                \item Inventroy
                \item Fight
                \item Challenge
			\end{packed_item}


		
			\subsection{Opis tablica}
			

				\textbf{User}   Ovaj entitet sadržava sve važne informacije o korisniku aplikacije, ali je apstraktan. Sadrži atribute: userID, username, name, email, profilePhoto, password, confirmed. Ovaj entitet je povezan s entitetima Player i Cartographer koji su specijalizacije entiteta User.
				
				
				\begin{longtblr}[
					label=none,
					entry=none
					]{
						width = \textwidth,
						colspec={|X[6,l]|X[6, l]|X[20, l]|}, 
						rowhead = 1,
					} %definicija širine tablice, širine stupaca, poravnanje i broja redaka naslova tablice
					\hline \multicolumn{3}{|c|}{\textbf{User}}	& \textbf{Type} & \textbf{Description}\\ \hline[3pt]
					\SetCell{LightGreen}userID & INT & jedinstveni identifikator korisnika\\ \hline
					username & VARCHAR & korisničko ime\\ \hline 
					name & VARCHAR & ime i prezime korisnika\\ \hline 
					email & VARCHAR	& email korisnika\\ \hline
					profilePhoto & BYTEA & fotografija korisnika\\ \hline
					password & VARCHAR & lozinka korisnika\\ \hline
					confirmed & BOOLEAN & je li korisnik potvrđen putem emaila\\ \hline 
				\end{longtblr}
				
				
				\textbf{Cartographer}   Ovaj entitet sadržava sve važne informacije o korisniku aplikacije koji je kartograf. Sadrži atribute: userID, IBAN, document i verified. Ovaj entitet je specijalizacija entiteta User te je u vezi s entitetom Card preko atributa approvedByUserID.
				
				
				\begin{longtblr}[
					label=none,
					entry=none
					]{
						width = \textwidth,
						colspec={|X[6,l]|X[6, l]|X[20, l]|}, 
						rowhead = 1,
					} %definicija širine tablice, širine stupaca, poravnanje i broja redaka naslova tablice
					\hline \multicolumn{3}{|c|}{\textbf{Cartographer}}	& \textbf{Type} & \textbf{Description}\\ \hline[3pt]
					\SetCell{LightGreen}userID & INT & jedinstveni identifikator korisnika\\ \hline
					IBAN & VARCHAR & IBAN kartografa\\ \hline 
					document & BYTEA & fotografija osobne iskaznice kartografa\\ \hline 
					verified & BOOLEAN	& je li administrator odobrio kartografa\\ \hline
				\end{longtblr}
				
			
			\textbf{Player}   Ovaj entitet sadržava sve važne informacije o korisniku aplikacije koji je igrač. Sadrži atribute: userID,  advanced, eloScore, banned i playerLocation. Ovaj entitet je specijalizacija entiteta User te je u vezi \textit{One-to-Many} s entitetom Inventory preko atributa userID te je u vezi \textit{One-to-Many} s entitetom Card preko atributa authorUserID. Također je u vezi \textit{One-to-Many} s entitetom Challenge preko atributa challengerUserID, tj. victimUserID kao što je i u \textit{One-to-Many} vezi s entiteom Fight preko atributa player1UserID, tj. player2UserID.
				
				
				\begin{longtblr}[
					label=none,
					entry=none
					]{
						width = \textwidth,
						colspec={|X[6,l]|X[6, l]|X[20, l]|}, 
						rowhead = 1,
					} %definicija širine tablice, širine stupaca, poravnanje i broja redaka naslova tablice
					\hline \multicolumn{3}{|c|}{\textbf{Player}}	& \textbf{Type} & \textbf{Description}\\ \hline[3pt]
					\SetCell{LightGreen}userID & INT & jedinstveni identifikator korisnika\\ \hline
					advanced	& BOOLEAN & razina ovlasti igrača može biti ordinary ili advanced\\ \hline 
					eloScore & INT & ELO score igrača\\ \hline 
					banned & BOOLEAN	& je li igrač isključen iz igre\\ \hline
					playerLocation & VARCHAR & aktivna lokacija korisnika\\ \hline
					signedIn & BOOLEAN & je li korisnik ulogiran\\ \hline
				\end{longtblr}
			
			\textbf{Admin}   Ovaj entitet sadržava sve važne informacije o adminu. Sadrži atribute: adminID, username, name, email, password.
				
				
				\begin{longtblr}[
					label=none,
					entry=none
					]{
						width = \textwidth,
						colspec={|X[6,l]|X[6, l]|X[20, l]|}, 
						rowhead = 1,
					} %definicija širine tablice, širine stupaca, poravnanje i broja redaka naslova tablice
					\hline \multicolumn{3}{|c|}{\textbf{Admin}}	& \textbf{Type} & \textbf{Description}\\ \hline[3pt]
					\SetCell{LightGreen}adminID & INT & jedinstveni identifikator admina\\ \hline
					username	& VARCHAR & korisničko ime\\ \hline 
					name & VARCHAR & ime i prezime admina\\ \hline 
					email & VARCHAR	& email admina\\ \hline
					password & VARCHAR & lozinka admina\\ \hline
				\end{longtblr}
				
				
			\textbf{Card}   Ovaj entitet sadržava sve važne informacije o kartici. Sadrži atribute: cardID, cardLocation, locationPhoto, title, description, cardStatus, authorUserID, approvedByUserID. Ovaj entitet je u vezi \textit{One-to-Many} s entitetom Inventory preko atributa cardID te je u \textit{One-to-Many} vezi s entitetom Fight preko atributa cardIDxy gdje je x element {1, 2} i y je element {1, 2, 3}. Card je također u \textit{One-to-One} vezi s entitetom Player preko atributa authorUserID te je u \textit{One-to-One} vezi s entitetom Cartographer preko atributa authorUserID.
				
				
				\begin{longtblr}[
					label=none,
					entry=none
					]{
						width = \textwidth,
						colspec={|X[6,l]|X[6, l]|X[20, l]|}, 
						rowhead = 1,
					} %definicija širine tablice, širine stupaca, poravnanje i broja redaka naslova tablice
					\hline \multicolumn{3}{|c|}{\textbf{Card}}	& \textbf{Type} & \textbf{Description}\\ \hline[3pt]
					\SetCell{LightGreen}cardID & INT & šifra kartice\\ \hline
					cardLocation & VARCHAR & lokacija kartice\\ \hline 
					locationPhoto & BYTEA & fotografija kartice\\ \hline 
					title & VARCHAR	& naziv kartice\\ \hline
					description & VARCHAR & opis kartice\\ \hline
					cardStatus & ENUM & status kartice može biti: submitted, unclaimed, claimed, verified\\ \hline
					authorUserID & INT & koji igrač je predložio dodavanje nove kartice u igru\\ \hline 
					approvedByUserID & INT & koji kartograf je odobrio zahtjev za izradu nove kartice\\ \hline
				\end{longtblr}
				
				
			\textbf{Inventory}   Ovaj entitet sadržava sve važne informacije o karticama koje pojedini igrači posjeduju. Sadrži atribute: userID, cardID i strength. Ovaj entitet je u \textit{Many-to-One} vezi s entitetom Player preko atributa userID te u \textit{Many-to-One} vezi s entitetom Card preko atributa cardID.
				
				
				\begin{longtblr}[
					label=none,
					entry=none
					]{
						width = \textwidth,
						colspec={|X[6,l]|X[6, l]|X[20, l]|}, 
						rowhead = 1,
					} %definicija širine tablice, širine stupaca, poravnanje i broja redaka naslova tablice
					\hline \multicolumn{3}{|c|}{\textbf{Inventory}}	& \textbf{Type} & \textbf{Description}\\ \hline[3pt]
					\SetCell{LightGreen}userID & INT & identifikacijski broj korisnika koji posjeduje karticu\\ \hline
					cardID	& INT & šifra kartice\\ \hline 
					strength & INT & jačina kartice\\ \hline
				\end{longtblr}
				
				
			\textbf{Fight}   Ovaj entitet sadržava sve važne informacije o borbi između dva igrača. Sadrži atribute: fightID, player1UserID, player2UserID, cardID11, cardID12, cardID13, cardID21, cardID22, cardID23, points1, points2, fightTimestamp, challengeID. Ovaj entitet je u \textit{One-to-One} vezi s entitetom Challenge preko atributa challengeID. Također je u \textit{Many-to-One} vezi s entitetom Player preko atributa userID te je i u \textit{Many-to-One} vezi s entitetom Card preko atributa cardID.
				
				
				\begin{longtblr}[
					label=none,
					entry=none
					]{
						width = \textwidth,
						colspec={|X[6,l]|X[6, l]|X[20, l]|}, 
						rowhead = 1,
					} %definicija širine tablice, širine stupaca, poravnanje i broja redaka naslova tablice
					\hline \multicolumn{3}{|c|}{\textbf{Fight}}	& \textbf{Type} & \textbf{Description}\\ \hline[3pt]
					\SetCell{LightGreen}fightID & INT & šifra borbe\\ \hline
					player1UserID & INT & pobjednik borbe\\ \hline
					player2UserID & INT & gubitnik borbe\\ \hline
					cardID11 & INT & prva kartica igrana od strane prvog igrača\\ \hline
					cardID12 & INT & druga kartica igrana od strane prvog igrača\\ \hline 
					cardID13 & INT & treća kartica igrana od strane prvog igrača\\ \hline 
					cardID21 & INT & prva kartica igrana od strane drugog igrača\\ \hline 
					cardID22 & INT & druga kartica igrana od strane drugog igrača\\ \hline 
					cardID23 & INT & treća kartica igrana od strane drugog igrača\\ \hline 
					points1 & FLOAT	& broj bodova prvog igrača\\ \hline
					points2 & FLOAT & broj bodova drugog igrača\\ \hline
					fightTimestamp & TIMESTAMP & trenutak u kojem je provedena borba\\ \hline
					challengeID & INT & šifra izazova\\ \hline
				\end{longtblr}
				
				
			\textbf{Challenge}   Ovaj entitet sadržava sve važne informacije o izazovu između dva igrača. Sadrži atribute: challengeID, challengerUserID, victimUserID, challengeTimestamp, challengeStatus. OVaj entitet je u \textit{One-to-One} vezi s entitetom Fight preko atributa challengeID te je u vezi \textit{Many-to-One} s entitetom Player preko atributa challengerUserID, tj. victimUserID.
				
				
				\begin{longtblr}[
					label=none,
					entry=none
					]{
						width = \textwidth,
						colspec={|X[6,l]|X[6, l]|X[20, l]|}, 
						rowhead = 1,
					} %definicija širine tablice, širine stupaca, poravnanje i broja redaka naslova tablice
					\hline \multicolumn{3}{|c|}{\textbf{Challenge}}	& \textbf{Type} & \textbf{Description}\\ \hline[3pt]
					\SetCell{LightGreen}challengeID & INT & šifra izazova\\ \hline
					challengerUserID & INT & igrač koji šalje zahtjev za borbu\\ \hline 
					victimUserID & INT & igrač koji je izazvan na borbu\\ \hline
					challengeTimestamp & TIMESTAMP & trenutak u kojem je nastao izazov\\ \hline
					challengeStatus & ENUM & je li izazov accepted, rejected ili pendning\\ \hline
				\end{longtblr}
			
			\subsection{Dijagram baze podataka}
				\textit{ U ovom potpoglavlju potrebno je umetnuti dijagram baze podataka. Primarni i strani ključevi moraju biti označeni, a tablice povezane. Bazu podataka je potrebno normalizirati. Podsjetite se kolegija "Baze podataka".}
			
			\eject
			
			
		\section{Dijagram razreda}
		
			\textit{Potrebno je priložiti dijagram razreda s pripadajućim opisom. Zbog preglednosti je moguće dijagram razlomiti na više njih, ali moraju biti grupirani prema sličnim razinama apstrakcije i srodnim funkcionalnostima.}\\
			
			\textbf{\textit{dio 1. revizije}}\\
			
			\textit{Prilikom prve predaje projekta, potrebno je priložiti potpuno razrađen dijagram razreda vezan uz \textbf{generičku funkcionalnost} sustava. Ostale funkcionalnosti trebaju biti idejno razrađene u dijagramu sa sljedećim komponentama: nazivi razreda, nazivi metoda i vrste pristupa metodama (npr. javni, zaštićeni), nazivi atributa razreda, veze i odnosi između razreda.}\\
			
			\textbf{\textit{dio 2. revizije}}\\			
			
			\textit{Prilikom druge predaje projekta dijagram razreda i opisi moraju odgovarati stvarnom stanju implementacije}
			
			
			
			\eject
		
		\section{Dijagram stanja}
			
			
			\textbf{\textit{dio 2. revizije}}\\
			
			\textit{Potrebno je priložiti dijagram stanja i opisati ga. Dovoljan je jedan dijagram stanja koji prikazuje \textbf{značajan dio funkcionalnosti} sustava. Na primjer, stanja korisničkog sučelja i tijek korištenja neke ključne funkcionalnosti jesu značajan dio sustava, a registracija i prijava nisu. }
			
			
			\eject 
		
		\section{Dijagram aktivnosti}
			
			\textbf{\textit{dio 2. revizije}}\\
			
			 \textit{Potrebno je priložiti dijagram aktivnosti s pripadajućim opisom. Dijagram aktivnosti treba prikazivati značajan dio sustava.}
			
			\eject
		\section{Dijagram komponenti}
		
			\textbf{\textit{dio 2. revizije}}\\
		
			 \textit{Potrebno je priložiti dijagram komponenti s pripadajućim opisom. Dijagram komponenti treba prikazivati strukturu cijele aplikacije.}